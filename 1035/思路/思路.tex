%%%%%%%%%%%%%%%%%%%%%%%%%%%%%%%%%%%%%%%%%
% Short Sectioned Assignment
% LaTeX Template
% Version 1.0 (5/5/12)
%
% This template has been downloaded from:
% http://www.LaTeXTemplates.com
%
% Original author:
% Frits Wenneker (http://www.howtotex.com)
%
% License:
% CC BY-NC-SA 3.0 (http://creativecommons.org/licenses/by-nc-sa/3.0/)
%
%%%%%%%%%%%%%%%%%%%%%%%%%%%%%%%%%%%%%%%%%

%----------------------------------------------------------------------------------------
%	PACKAGES AND OTHER DOCUMENT CONFIGURATIONS
%----------------------------------------------------------------------------------------

\documentclass[ paper=a4,UTF8, fontsize=11pt]{ctexart}
\usepackage[top=1in, bottom=1in, left=1.25in, right=1.25in]{geometry}
\usepackage{clrscode}
\usepackage{booktabs}
\usepackage{enumerate}
\usepackage{float}
\usepackage{diagbox}
\usepackage{graphicx}
\usepackage{multirow}
\usepackage{CJK} % Use 8-bit encoding that has 256 glyphs
\usepackage{fourier} % Use the Adobe Utopia font for the document - comment this line to return to the LaTeX default
\usepackage[english]{babel} % English language/hyphenation
\usepackage{amsmath,amsfonts,amsthm} % Math packages
\usepackage{amssymb}
\usepackage{lipsum} % Used for inserting dummy 'Lorem ipsum' text into the template
\usepackage{bm}
\usepackage{listings}
\usepackage[framed,numbered,autolinebreaks,useliterate]{mcode}


\usepackage{fancyhdr} % Custom headers and footers
\pagestyle{fancyplain} % Makes all pages in the document conform to the custom headers and footers
\fancyhead{} % No page header - if you want one, create it in the same way as the footers below
\fancyfoot[L]{} % Empty left footer
\fancyfoot[C]{} % Empty center footer
\fancyfoot[R]{\thepage} % Page numbering for right footer
\renewcommand{\headrulewidth}{0pt} % Remove header underlines
\renewcommand{\footrulewidth}{0pt} % Remove footer underlines
\setlength{\headheight}{13.6pt} % Customize the height of the header

\numberwithin{equation}{section} % Number equations within sections (i.e. 1.1, 1.2, 2.1, 2.2 instead of 1, 2, 3, 4)
\numberwithin{figure}{section} % Number figures within sections (i.e. 1.1, 1.2, 2.1, 2.2 instead of 1, 2, 3, 4)
\numberwithin{table}{section} % Number tables within sections (i.e. 1.1, 1.2, 2.1, 2.2 instead of 1, 2, 3, 4)

\setlength\parindent{0pt} % Removes all indentation from paragraphs - comment this line for an assignment with lots of text

%----------------------------------------------------------------------------------------
%	TITLE SECTION
%----------------------------------------------------------------------------------------
\begin{document}
\begin{titlepage}
\newcommand{\horrule}[1]{\rule{\linewidth}{#1}} % Create horizontal rule command with 1 argument of height

\title{	
\normalfont \normalsize
\textsc{School of Software, Tsinghua University} \\ [25pt] % Your university, school and/or department name(s)
\horrule{0.5pt} \\[0.4cm] % Thin top horizontal rule
\huge dp \\ % The assignment title
\horrule{2pt} \\[0.5cm] % Thick bottom horizontal rule
}


\author{孟骋} % Your name
\date{\normalsize\today} % Today's date or a custom date
\maketitle % Print the title
\end{titlepage}



%----------------------------------------------------------------------------------------
%	PROBLEM 1
%----------------------------------------------------------------------------------------
\newpage
\begin{CJK}{UTF8}{song}
使用dp[i][j]表示已经询问了子树的所有关键节点,人车的一个状态,其中:
\begin{itemize}
  \item j==0:人去,不管人是否回来
  \item j==1:人去,人一定要回来
  \item j==2:人车都去,人车都要回来
  \item j==3:人车都去,人一定要回来,车不管
  \item j==4:人车都去,不管人车最后是否回来
\end{itemize}
这里显然有dp[i][1]>=dp[i][0],dp[i][2]>=dp[i][3]>=dp[i][4] \par
假设某个结点为fa,其某一子节点为son,w1(fa,son)表示步行的代价,w2(fa,son)表示开车的代价,则有:
\begin{displaymath}
  dp[fa][1]=\sum \left( dp[son][1] + 2*w1(fa,son)  \right)
\end{displaymath}
即遍历所有的子节点(w1(fa,son)),然后遍历子节点的所有子节点,并且回到子节点(dp[son][1]),然后从子节点回来(w1(fa,son))。 注意,这是一棵树,不存在环。  \par
\begin{displaymath}
  dp[fa][2]=\sum \left( \min(2*w1(fa,son)+dp[son][1] ,dp[son][2]+2*w2(fa,son))    \right)
\end{displaymath}
分为两种情况,一种是人步行去,然后步行回来,第二种是人车都去,人车都回来。  \par
\begin{flalign}
  temp &=\min(dp[son][0]-dp[son][1]-w1(fa,son)  )  \nonumber \\
  temp &=\min(temp,0)  \nonumber \\
  dp[fa][0] &=dp[fa][1]+temp  \nonumber 
\end{flalign}
分两种情况:
\begin{itemize}
  \item 人不回来,一个边选择为dp[son][0]+w1(fa,son),其它的都选择dp[son][1]+2*w1(fa,son)
  \item 人回来,  所有边都为dp[son][1]+2*w1(fa,son)=dp[fa][1]
\end{itemize}
\begin{flalign}
  t2 &=\min(dp[son][1]+2*w1(fa,son) ,dp[son][2]+2*w2(fa,son) ) =dp[fa][2]  \nonumber\\
  t3 &=0  \nonumber \\
  t3 &=\min(t3,w1(fa,son)+w2(fa,son)+dp[son][3]-t2 )   \nonumber \\
  dp[fa][3] &=dp[fa][2]+t3              \nonumber
\end{flalign}
dp[fa][3]的决策方式:
\begin{itemize}
  \item 选择一个子节点开车去,然后步行回来(w2(fa,son)),其它的选择t2.
  \item 都选择t2
\end{itemize}
等价于:
\begin{displaymath}
  dp[fa][3]=\min(dp[fa][2],  dp[fa][2] -t2+ w1(fa,son)+w2(fa,son)+dp[son][3])
\end{displaymath}
最后是dp[fa][4],从走的最后一棵子树考虑:
\begin{itemize}
  \item 走最后一棵子树时,还有车:最后一棵子树的选择是 $\min(w1(fa,son)+dp[son][0],w2(fa,son)+dp[son][4] )$,其它选择都是t2.或者全部都选择t2.
  \item 走最后一棵子树时,没有车了。则最后一次为w1(fa,son)+dp[son][0],其余访问中,必定有一次人把车开走,且没开回来,这次是总的倒数第二次,即:w2(fa,son)+dp[son][3]+w1(fa,son),其余均为
  t2.若最后人回来了,则退化为dp[fa][3].因为有两棵特殊的子树,类似上面的差值统计的时候,要避免两个差值代表同一个子树。\par
  记:
  \begin{flalign}
    ff1 &=w1(fa,son)+w2(fa,son)+dp[son][3]-t2      \nonumber \\
    ff2 &=w1(fa,son)+dp[son][0]-t2     \nonumber
  \end{flalign}
  目标是找到不同son,使得ff1+ff2最小。可以先去找ff1的最小和次小,并记录达到最小ff1的儿子match.再遍历一下所有儿子,如果此时的son就是match,那么配合ff1的次小值,否则配合最小值。
\end{itemize}
%----------------------------------------------------------------------------------------
\end{CJK}

\end{document}
